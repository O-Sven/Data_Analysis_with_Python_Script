%%%% SETUP %%%%

\documentclass[11pt, a4paper]{article}
\voffset=-2.5cm
\hoffset=-2cm
\textwidth=16.8cm
\textheight=26.15cm
\marginparwidth=0cm
\marginparsep=0cm

%\usepackage{showframe}
\usepackage[utf8]{inputenc}
\usepackage{graphicx}

\setlength\parindent{0pt}
\renewcommand{\baselinestretch}{1.2} 


%%%% MAIN DOCUMNET %%%%

\begin{document}

\begin{center}
\thispagestyle{empty}
\includegraphics[scale=0.7]{siegel_buw.eps}
\vspace{0.5cm}

\huge Teilnahmebescheinigung \\[-0.5cm]
\rule{\linewidth}{0.2pt}\\[0.2cm]
{\Huge \textbf{Datenanalyse mit Python}}\\[-0.4cm]
\rule{\linewidth}{0.2pt}\\[1cm]
\Large
Die Bergische Universität Wuppertal bestätigt hiermit\\[0.5cm]
\textbf{\LARGE PLATZHALTER-NAMEN}\\[0.5cm]
die Teilnahme am Kurs\\
\textbf{Datenanalyse mit Python}\\
an der Fakultät für Architektur und Bauingenieurwesen.\\
(Umfang: 11 Termine à 90 Minuten)\\
\vspace{1cm}
\large
Themengebiete:
\begin{itemize}
\setlength\itemsep{0pt}
\item Grundlagen der Programmierung auf Basis von \textit{Python3}
\item Automatisiertes Ein- und Auslesen von Datensätzen
\item Datenvisualisierung unter Verwendung von \textit{matplotlib}
\item Datenanalyse mittels \textit{NumPy} (Bisektion-Verfahren, Datenglättung) 
\item Datenoptimierungsverfahren mittels \textit{NumPy} und \textit{SciPy}\\
(Parameterschätzung und -optimierung)
\item Bildverarbeitung (Bildsequenzen mittels \textit{ffmpeg}, Verfolgung von Objekten) 
\end{itemize}
\end{center}

\vspace{1.3cm}
Wuppertal, den 08. Februar 2019\\[1.2cm]
%
...........................................\\
Dr. rer. nat. Lukas Arnold




\end{document}
